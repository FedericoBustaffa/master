\documentclass[12pt, a4paper]{article}

\usepackage[T1]{fontenc}
\usepackage[english]{babel}
\usepackage{mathtools, amsmath, amssymb, amsthm}
\usepackage[hidelinks]{hyperref}
\usepackage{tabularx}
\usepackage{svg}
\usepackage{caption}
\usepackage{float}
\usepackage{booktabs}

\title{Assignment 1\\Intrinsics and Auto-Vectorization}
\author{Federico Bustaffa}
\date{12/03/2025}

\begin{document}

\maketitle
\tableofcontents

\section{Introduction}

This report discuss the methodology adopted for the optimized implementations
of the \textit{softmax} function. The first one using the \textit{auto-vectorization}
provided by the compiler and the second using explicit vectorization with
\textit{intrinsics}.

Both versions were compared to each other and with the \textit{plain} naive
version provided by the teacher, mainly evaluating the execution time as the
input size varies.

\section{Methods and Implementation}

The plain version of the algorithm has been compiled with only the \verb|-O2|
flag, disabling the \textit{auto-vectorization}; mainly to see the improvement
over a non vectorized version.

For both versions, optimizations were applied incrementally by adding changes
and testing performance at each step.

\subsection{Auto-Vectorization}

To check if optimizations were applied by the compiler, it has been used the
\verb|-fopt-info-vec-missed| flag.

The first step was to use \verb|-O3| flag to enable all possible optimizations
the compiler can apply on the plain version of the algorithm.

The first step of the algorithm finds the maximum value of the given array, by
looping over it. To improve this step the was applied the \textit{loop unrolling}
technique by a factor of 2 and 4.

\begin{table}[H]
	\centering
	\begin{tabular}{rrrr}
		\toprule
		Elements & Plain($\mu$s) & Unroll2($\mu$s) & Unroll4($\mu$s) \\
		\midrule
		128      & 6.0034        & 6.1522          & 5.4836          \\
		1024     & 16.0794       & 12.7019         & 10.4904         \\
		8192     & 98.5208       & 42.4009         & 36.0347         \\
		16384    & 187.9920      & 74.9148         & 67.2819         \\
		\bottomrule
	\end{tabular}
	\caption{Execution time for different factors of loop unrolling}
\end{table}

\subsection{Intrinsics}

\section{Comparisons}


\end{document}

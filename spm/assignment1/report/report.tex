\documentclass[12pt, a4paper]{article}

\usepackage[T1]{fontenc}
\usepackage[english]{babel}
\usepackage{mathtools, amsmath, amssymb, amsthm}
\usepackage[hidelinks]{hyperref}
\usepackage{tabularx}
\usepackage{svg}
\usepackage{caption}
\usepackage{float}
\usepackage{booktabs}

\title{Assignment 1\\Intrinsics and Auto-Vectorization}
\author{Federico Bustaffa}
\date{12/03/2025}

\begin{document}

\maketitle
\tableofcontents

\section{Introduction}

This report discuss implementation choices and performances of the two
implementations of the \textit{softmax} function. The first using the
\textit{auto-vectorization} provided by the compiler and the second using
explicit vectorization with \textit{intrinsics}.

Both versions were compared to each other and with the plain naive version
provided by the teacher, mainly evaluating the execution time as the input size
varies.

\section{Implementation}

For both versions, optimizations optimizations were applied incrementally,
adding changes and testing performance at each step. The performance evaluation
was not fine grained, it always measures the global execution time that the
function takes.

\subsection{Auto-Vectorization}

To check if optimizations were applied by the compiler, it has been used the
\verb|-fopt-info-vec-missed| flag.

The first step of the algorithm finds the maximum value of the given array, by
looping over it. To improve this step the was applied the \textit{loop unrolling}
technique by a factor of 2 and 4.

\begin{table}[H]
	\centering
	\begin{tabular}{lrrr}
		\toprule
		Elements & Plain(ms) & Unroll2 Time(ms) & Unroll4 Time(ms) \\
		\midrule
		128      & 0.00674   & 0.00593          & 0.00697          \\
		1024     & 0.01639   & 0.01244          & 0.01238          \\
		8192     & 0.09465   & 0.04024          & 0.04011          \\
		16384    & 0.18654   & 0.07079          & 0.07320          \\
		\bottomrule
	\end{tabular}
\end{table}

\subsection{Intrinsics}

\section{Comparisons}


\end{document}
